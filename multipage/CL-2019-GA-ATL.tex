
I am writing in application to the GA opening in the Accessible Technologies Lab providing support and training with assistive technologies to interested students, staff, and faculty to aid in their time at the university. I believe the combination of my engineering work implementing accessibility-oriented technologies, my teaching work overhauling an already difficult course with a focus on accessibility, and the ever-growing list of trainings and bootcamps focused on accessibility and facilitation all make me a top-tier choice for the position.

The bulk of my past engineering work exemplifies my adaptability and versatility in addressing an unmet need via whatever tools and technologies provide the best experience for the user. My research experience includes developing a non-invasive Brain-Computer Interface that doesn't hurt to use, in-the-moment fall risk prediction using motion capture, and developing an accessibility mapping infrastructure to provide relevant information both to disabled individuals attempting to navigate campus, and to various administrative offices interested in improving capus accessibility. I am primary inventor behind two patents on augmented reality BCI interfaces for device control in situations preventing the use of hands or voice control. In addition, I've significant experience across the full software/hardware stack and the breadth of operating systems the ATL's clientel may be using, so can both adapt and debug whatever situation may emerge.

As an educator, I have drastically overhauled a two-course sequence to improve course accessibility/approachability across the wide varieties of neurotypes, financial backgrounds, and educational backgrounds of my students. Within my first semester as a TA, I convinced the main instructors to drop tests outright due to the multitude of barriers they impose to student learning. In the subsequent three semesters as the de-facto course instructor, I began recording lectures and uploading them on canvas for students to be able to watch from home and on slower or faster speeds; I of course also used the ATL-provided automatic captioning system as well, and taught my co-instructor to do the same. I adopted an approachable, less-than-formal instructional style and added explicitly non-course-related "advisory" hours to cultivate an atmosphere of mutual respect that the students responded to fantastically. 

I am exceedingly excited at the prospect of tackling the problem of accessibility from a user-centric perspective, rather than as the engineer of the tools or the teacher behind the potential problems. As an autistic adult, I often find myself wishing there were offices like the ATL that I could go to for support, and I would be overjoyed to get to provide that support with assistive technologies that students, staff, and faculty need here at Virginia Tech.
